% Copyright 2019 the authors. All rights reserved.

% TODO:
% -

\documentclass[modern]{aastex62}

\usepackage{amsmath}

% typography
\setlength{\parindent}{1.\baselineskip}
\newcommand{\acronym}[1]{{\small{#1}}}
\newcommand{\package}[1]{\textsl{#1}}
\newcommand{\gaia}{\textsl{Gaia}}
\newcommand{\pans}{\textsl{Pan-STARRS}}
\newcommand{\DR}{\acronym{DR2}}
\newcommand{\msun}{\textrm{M}_\odot}
\newcommand{\kpc}{\textrm{kpc}}
\newcommand{\kms}{\ensuremath{\textrm{km}~\textrm{s}^{-1}}}
\newcommand{\bs}[1]{\boldsymbol{#1}}
\newcommand{\masyr}{\ensuremath{\textrm{mas}~\textrm{yr}^{-1}}}
\newcommand{\feh}{\ensuremath{[\textrm{Fe} / \textrm{H}]}}
\newcommand{\given}{\,|\,}

\newcommand{\sectionname}{Section}
\newcommand{\equationname}{Equation}
\renewcommand{\tablename}{Table}

\newcommand{\todo}[1]{{\color{red} TODO: #1}}

\newcommand{\changes}[1]{{\textbf{#1}}}
% \newcommand{\changes}[1]{{#1}}

% aastex parameters
% \received{not yet; THIS IS A DRAFT}
%\revised{not yet}
%\accepted{not yet}
% % Adds "Submitted to " the arguement.
% \submitjournal{ApJ}
\shorttitle{Stuff}
\shortauthors{People}

%@arxiver{}

\begin{document}\sloppy\sloppypar\raggedbottom\frenchspacing % trust me

\title{Flashy title}

% \author[0000-0003-0872-7098]{Adrian~M.~Price-Whelan}
% \affiliation{Department of Astrophysical Sciences,
%              Princeton University, Princeton, NJ 08544, USA}
% \email{adrn@astro.princeton.edu}
% \correspondingauthor{Adrian M. Price-Whelan}

% \author[0000-0002-7846-9787]{Ana Bonaca}
% \affiliation{Harvard--Smithsonian Center for Astrophysics, Cambridge, MA 02138, USA}

\begin{abstract}\noindent % trust me
    Words!
\end{abstract}

\keywords{Galaxy: halo --- dark matter ---
          Galaxy: kinematics and dynamics}

\section{Introduction}
\label{sec:intro}


\section{Data}

\section{Density model}
We first transform our  simulated Pal 5 data points to the tangent sky plane using a Zeanit (Lambert azimuthal) equal-area projection, such that we can define a Gaussian. We call these coordinates, $X$, $Y$.

We then fit a 3rd order polynomial to the leading and trailing arm separately, in this projected space. 

We place K nodes, k, equally in distance along the polynomial fits to the leading and trailing arm. 

At each node, k, along the polynomial we find the tangent/parallel unit vector,  $\hat{u}$, and the perpendicular normal unit vector, $\hat{v}$, to the stream. 

At each node, k, we define the co-variance matrix, $\tilde{C_k}$:

\begin{equation}
\tilde{C_k} = 
\begin{pmatrix}
    h^2 & 0  \\
    0 & S_k^2  \\
\end{pmatrix}
\end{equation}
where $h$ is bandwidth of the Gaussian components along the polynomial fit and $S_k$ is the  width of the Gaussians in the perpendicular (normal) direction of the stream at any node, k. 

We transform it from the space spanned by the $\hat{u}$,  $\hat{v}$ vectors to $X$, $Y$:
\begin{equation}
C_k = \rm{R} \tilde{C_k} \rm{R}^T
\end{equation}
where
\begin{equation}
\rm{R} = 
\begin{pmatrix}
    \rm{cos} \theta & - \rm{cos} \theta  \\
    \rm{cos} \theta & \rm{cos} \theta \\
\end{pmatrix}
\end{equation}
and  $\theta$ is the angle between the tangent sky plane and the unit vector, $\hat{u}$.

To compute our density model along the leading at trailing stream, for the K nodes, k, we define ln density  and sum over the K nodes:
\begin{equation}
\rm{ln} \sum ( \alpha_k \mathcal{N}\left(\mu_k, C_k\right))
\end{equation}
where $\mu_k$ denotes the location of each node, k, along the leading and trailing polynomial fits, respectively, $C_k$ is the covariance matrix, and $\alpha_k$ is the amplitude of the Gaussian (representing density at specific node, k). 

We then fit for the perpendicular width, $S_k$, to the stream (polynomial fits) and the amplitude of the Gaussians, $\alpha_k$, which represent the density along the stream. 

%Explain background model for data.

We compute the width and density of both or simulated Pal 5 streams and the DECaLS data fitting the above Gaussian mixture model. 




\acknowledgements{
It is a pleasure to thank


This work has made use of data from the European Space Agency (ESA) mission {\it
Gaia} (\url{https://www.cosmos.esa.int/gaia}), processed by the {\it Gaia} Data
Processing and Analysis Consortium (DPAC,
\url{https://www.cosmos.esa.int/web/gaia/dpac/consortium}). Funding for the DPAC
has been provided by national institutions, in particular the institutions
participating in the {\it Gaia} Multilateral Agreement.  This research was
started at the NYC Gaia DR2 Workshop at the Center for Computational
Astrophysics of the Flatiron Institute in 2018 April.

AB acknowledges generous support from the Institute for Theory and Computation
at Harvard University.
% All code used in this work and all results are available at
% \url{https://github.com/adrn/GD1-DR2}.
}

\software{
    \package{Astropy} \citep{astropy, astropy:2018},
    \package{dustmaps}\footnote{\url{https://github.com/gregreen/dustmaps}},
    \package{gala} \citep{gala},
    \package{IPython} \citep{ipython},
    \package{matplotlib} \citep{mpl},
    \package{numpy} \citep{numpy},
    \package{scipy} \citep{scipy}
}

\bibliographystyle{aasjournal}
\bibliography{pal5fan}

\end{document}
