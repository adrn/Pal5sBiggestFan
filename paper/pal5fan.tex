% Copyright 2019 the authors. All rights reserved.

% TODO:
% -

\documentclass[modern]{aastex62}

\usepackage{amsmath}

% typography
\setlength{\parindent}{1.\baselineskip}
\newcommand{\acronym}[1]{{\small{#1}}}
\newcommand{\package}[1]{\textsl{#1}}
\newcommand{\gaia}{\textsl{Gaia}}
\newcommand{\pans}{\textsl{Pan-STARRS}}
\newcommand{\DR}{\acronym{DR2}}
\newcommand{\msun}{\textrm{M}_\odot}
\newcommand{\kpc}{\textrm{kpc}}
\newcommand{\kms}{\ensuremath{\textrm{km}~\textrm{s}^{-1}}}
\newcommand{\bs}[1]{\boldsymbol{#1}}
\newcommand{\masyr}{\ensuremath{\textrm{mas}~\textrm{yr}^{-1}}}
\newcommand{\feh}{\ensuremath{[\textrm{Fe} / \textrm{H}]}}
\newcommand{\given}{\,|\,}

\newcommand{\sectionname}{Section}
\newcommand{\equationname}{Equation}
\renewcommand{\tablename}{Table}

\newcommand{\todo}[1]{{\color{red} TODO: #1}}

\newcommand{\changes}[1]{{\textbf{#1}}}
% \newcommand{\changes}[1]{{#1}}

% aastex parameters
% \received{not yet; THIS IS A DRAFT}
%\revised{not yet}
%\accepted{not yet}
% % Adds "Submitted to " the arguement.
% \submitjournal{ApJ}
\shorttitle{Stuff}
\shortauthors{People}

%@arxiver{}

\begin{document}\sloppy\sloppypar\raggedbottom\frenchspacing % trust me

\title{Flashy title}

% \author[0000-0003-0872-7098]{Adrian~M.~Price-Whelan}
% \affiliation{Department of Astrophysical Sciences,
%              Princeton University, Princeton, NJ 08544, USA}
% \email{adrn@astro.princeton.edu}
% \correspondingauthor{Adrian M. Price-Whelan}

% \author[0000-0002-7846-9787]{Ana Bonaca}
% \affiliation{Harvard--Smithsonian Center for Astrophysics, Cambridge, MA 02138, USA}

\begin{abstract}\noindent % trust me
    Words!
\end{abstract}

\keywords{Galaxy: halo --- dark matter ---
          Galaxy: kinematics and dynamics}

\section{Introduction}
\label{sec:intro}


\section{Data}
\label{sec:data}

\section{Simulations}
\label{sec:sim}
In order to explore the mechanism leading the the observed morphology of the stream (e.g. the length asymmetry between the leading and trailing arm, as well as the gap in the trailing arm, and the ``fan" in the leading arm), we run a suite of Pal 5 simulations. In particular, we investigate whether Pal 5's ``fan" can be explained from mild chaos  (\citealt{Pearson:2015}, \citealt{Price-Whelan:2016}), or whether Pal 5 is interacting with the Galactic bar (\citealt{Pearson:2017}, \citealt{Erkal:2017}, \citealt{Banik:2019}). In Section \ref{sec:potential}, we describe the potentials in which we simulate the evolution of Pal 5, and in Section \ref{sec:potential} we describe our suite of Pal 5 stream simulations. 

\subsection{Potential}
\label{sec:potential}
We simulate the evolution of Pal 5 in two classes of three-component Galactic potentials: 

\begin{itemize}
\item[1.] {\bf Static potential}: we use the {\small MWPotential2014} (\citealt{Bovy:2015}) consisting of a Miyamoto-Nagai disk (\citealt{Miyamoto:1975}), a bulge modeled as an exponentially cut off, power-law density profile, and an NFW dark matter halo (\citealt{Navarro:1996}). We vary the flattening of this halo to demonstrate both a regular and chaotic Pal 5 orbit (see Section \ref{sec:modeling}). 

\item[2.] {\bf  Barred potential}: we use the same disk and halo as in {\small MWPotential2014}, but include a Galactic bar instead of a bulge. Following \citet{wang:2012}, we compute the bar potential as a basis-function expansion (BFE) representation of a triaxial, exponential density profile:

\begin{equation}
\rho_{bar} = \rho_0 [{\rm exp} (-r^2_1/2) + r_2^{-1.85} {\rm exp}(-r_2) ]
\end{equation}

\begin{equation}
r_1 = \left[((x/x_0)^2 + (y/y_0)^2)^2 +( z/z_0)^4\right]^{1/4}
\end{equation}

\begin{equation}
r_2 = \left[\frac{q^2(x^2 + y^2) + z^2)}{z_0^2}\right]^{1/2}
\end{equation}
where the scale length is $x_0$ = 1.49 kpc, $y_0$ = 0.58 kpc, $z_0$ = 0.4 kpc, and q = 0.6. We include terms up to $n=9$, $l=19$ in the ``self-consistent field" BFE formalism, as this yields a good representation of the density of the bar (\citealt{Banik:2019})\footnote{Note that using lower order terms (e.g. n=6, l = 8) for the basis function expansion does not much change the morphology or kinematics of the Pal 5 stream.}. We explore barred models with pattern speeds of $\Omega_b$ = ($25 - 65$) $\kms$ kpc$^{-1}$ in increments of 1 kpc$^{-1}$ and bar masses of $M_{bar} = 5 \times 10^{9}$ $\msun$ and $M_{bar} = 1 \times 10^{10}$ $\msun$.
\end{itemize}

In \citet{wang:2012}, they construct a bar with a pattern speed of $\Omega_b$ =  60 $\kms$ kpc$^{-1}$. This corresponds to a barred model with a co-rotation at $r_{\rm CR} = 3.2$ kpc. However, in this work we explore a range of pattern speeds. As bars are not expected to extend much beyond their co-rotation radius (\citealt{binney:2008}), we adjust the physical scaling of the bar when varying the pattern speed. %In \citet{wang:2012}, the scale-radius, $r_s$, assumed in the BFE, is  $r_s = 1.1$ kpc. 
To include the fact that we are using various pattern speeds in our bar models, we  compute the co-rotation radius, $r_{\rm CR},$ based on the mass profiles of the disk and bulge (from the static potential) at any given pattern speed, $\Omega_b$. We then scale our bar model for a given pattern speed, $\Omega_b$, by:

\begin{equation} 
r_{s, \Omega_b}  = r_{{\rm CR}, \Omega_b}/r_{{\rm CR, Wang 2012}}
\end{equation} 

If this scaling parameter of the bar is left unaltered (e.g. for the bar models used in \citealt{price:2016b}, \citealt{Pearson:2017}, \citealt{Erkal:2017}, \citealt{Banik:2019}), this creates a too strong a bar quadrupole for the faster pattern speeds, and too weak bar quadrupole for the slower pattern speeds. 

In Figure \ref{fig:vcirc}, we show the circular velocity curve for our barred, three-component Galactic potential for the range of  $\Omega_b$ = ($25 - 65$) $\kms$ kpc$^{-1}$ and for the two different bar masses explored in this paper (red: $M_{bar} = 5 \times 10^{9}$ $\msun$, blue: $M_{bar} = 1 \times 10^{10}$ $\msun$).

\begin{figure}
\centerline{\includegraphics[width=\columnwidth]{v_circ_nlm919.png}}
\caption{\todo{Probably include static potential as well, and show the different components of the potentials }}
\label{fig:vcirc}
\end{figure}



\subsection{Stream modeling}
\label{sec:modeling}
For all simulations we assume a 6D phase space position of Pal 5 as: RA = 229.018 deg, Dec = - 0.24 deg, distance = 22.9 kpc, $v_r$ = -58.7 , pm$_{RA,cosdec}= -2.296$ mas and pm$_{Dec} = -2.257$ mas. We first transform Pal 5's 6D phase space position into a Galactocentric frame by assuming $v_{lsr} = (11.1, 24.0, 7.25) \kms$, 
$v_{circ} = 220  \kms$ and a distance from the Sun to the Galactic centre of 8.1 kpc. 

We first integrate the cluster backwards in time for 4 Gyr in steps of 0.5 Myr. Subsequently, simulate the evolution of Pal 5 using the ``particle-spray" stream modeling method developed by \citet{Fardal:2015}. We release two particles through each of the two Lagrange points every 10 Myr. We repeat this in first the NFW potential setting $q_z = 0.94$ (\citealt{Bovy:2016}) and subsequently in a flattened NFW potential ($q_z = 0.5$ to induce a chaotic orbit). 

Additionally, we simulate the evolution of Pal 5's stream in the barred potential




\section{Density model}
We first transform our  simulated Pal 5 data points to the tangent sky plane using a Zeanit (Lambert azimuthal) equal-area projection, such that we can define a Gaussian. We call these coordinates, $X$, $Y$.

We then fit a 3rd order polynomial to the leading and trailing arm separately, in this projected space. 

We place K nodes, k, equally in distance along the polynomial fits to the leading and trailing arm. 

At each node, k, along the polynomial we find the tangent/parallel unit vector,  $\hat{u}$, and the perpendicular normal unit vector, $\hat{v}$, to the stream. 

At each node, k, we define the co-variance matrix, $\tilde{C_k}$:

\begin{equation}
\tilde{C_k} = 
\begin{pmatrix}
    h^2 & 0  \\
    0 & S_k^2  \\
\end{pmatrix}
\end{equation}
where $h$ is bandwidth of the Gaussian components along the polynomial fit and $S_k$ is the  width of the Gaussians in the perpendicular (normal) direction of the stream at any node, k. 

We transform it from the space spanned by the $\hat{u}$,  $\hat{v}$ vectors to $X$, $Y$:
\begin{equation}
C_k = \rm{R} \tilde{C_k} \rm{R}^T
\end{equation}
where
\begin{equation}
\rm{R} = 
\begin{pmatrix}
    \rm{cos} \theta & - \rm{cos} \theta  \\
    \rm{cos} \theta & \rm{cos} \theta \\
\end{pmatrix}
\end{equation}
and  $\theta$ is the angle between the tangent sky plane and the unit vector, $\hat{u}$.

To compute our density model along the leading at trailing stream, for the K nodes, k, we define ln density  and sum over the K nodes:
\begin{equation}
\rm{ln} \sum ( \alpha_k \mathcal{N}\left(\mu_k, C_k\right))
\end{equation}
where $\mu_k$ denotes the location of each node, k, along the leading and trailing polynomial fits, respectively, $C_k$ is the covariance matrix, and $\alpha_k$ is the amplitude of the Gaussian (representing density at specific node, k). 

We then fit for the perpendicular width, $S_k$, to the stream (polynomial fits) and the amplitude of the Gaussians, $\alpha_k$, which represent the density along the stream. 

%Explain background model for data.

We compute the width and density of both or simulated Pal 5 streams and the DECaLS data fitting the above Gaussian mixture model. 

\section{Discussion}



\acknowledgements{
It is a pleasure to thank


This work has made use of data from the European Space Agency (ESA) mission {\it
Gaia} (\url{https://www.cosmos.esa.int/gaia}), processed by the {\it Gaia} Data
Processing and Analysis Consortium (DPAC,
\url{https://www.cosmos.esa.int/web/gaia/dpac/consortium}). Funding for the DPAC
has been provided by national institutions, in particular the institutions
participating in the {\it Gaia} Multilateral Agreement.  This research was
started at the NYC Gaia DR2 Workshop at the Center for Computational
Astrophysics of the Flatiron Institute in 2018 April.

AB acknowledges generous support from the Institute for Theory and Computation
at Harvard University.
% All code used in this work and all results are available at
% \url{https://github.com/adrn/GD1-DR2}.
}

\software{
    \package{Astropy} \citep{astropy, astropy:2018},
    \package{dustmaps}\footnote{\url{https://github.com/gregreen/dustmaps}},
    \package{gala} \citep{gala},
    \package{IPython} \citep{ipython},
    \package{matplotlib} \citep{mpl},
    \package{numpy} \citep{numpy},
    \package{scipy} \citep{scipy}
}

\bibliographystyle{aasjournal}
\bibliography{pal5fan}

\end{document}
