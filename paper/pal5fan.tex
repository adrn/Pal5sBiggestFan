% Copyright 2019 the authors. All rights reserved.

% TODO:
% -

\documentclass[modern]{aastex62}

\usepackage{amsmath}

% typography
\setlength{\parindent}{1.\baselineskip}
\newcommand{\acronym}[1]{{\small{#1}}}
\newcommand{\package}[1]{\textsl{#1}}
\newcommand{\gaia}{\textsl{Gaia}}
\newcommand{\pans}{\textsl{Pan-STARRS}}
\newcommand{\DR}{\acronym{DR2}}
\newcommand{\msun}{\textrm{M}_\odot}
\newcommand{\kpc}{\textrm{kpc}}
\newcommand{\kms}{\ensuremath{\textrm{km}~\textrm{s}^{-1}}}
\newcommand{\bs}[1]{\boldsymbol{#1}}
\newcommand{\masyr}{\ensuremath{\textrm{mas}~\textrm{yr}^{-1}}}
\newcommand{\feh}{\ensuremath{[\textrm{Fe} / \textrm{H}]}}
\newcommand{\given}{\,|\,}

\newcommand{\sectionname}{Section}
\newcommand{\equationname}{Equation}
\renewcommand{\tablename}{Table}

\newcommand{\todo}[1]{{\color{red} TODO: #1}}

\newcommand{\changes}[1]{{\textbf{#1}}}
% \newcommand{\changes}[1]{{#1}}

% aastex parameters
% \received{not yet; THIS IS A DRAFT}
%\revised{not yet}
%\accepted{not yet}
% % Adds "Submitted to " the arguement.
% \submitjournal{ApJ}
\shorttitle{Stuff}
\shortauthors{People}

%@arxiver{}

\begin{document}\sloppy\sloppypar\raggedbottom\frenchspacing % trust me

\title{Flashy title}

% \author[0000-0003-0872-7098]{Adrian~M.~Price-Whelan}
% \affiliation{Department of Astrophysical Sciences,
%              Princeton University, Princeton, NJ 08544, USA}
% \email{adrn@astro.princeton.edu}
% \correspondingauthor{Adrian M. Price-Whelan}

% \author[0000-0002-7846-9787]{Ana Bonaca}
% \affiliation{Harvard--Smithsonian Center for Astrophysics, Cambridge, MA 02138, USA}

\begin{abstract}\noindent % trust me
    Words!
\end{abstract}

\keywords{Galaxy: halo --- dark matter ---
          Galaxy: kinematics and dynamics}

\section{Introduction}
\label{sec:intro}


\acknowledgements{
It is a pleasure to thank


This work has made use of data from the European Space Agency (ESA) mission {\it
Gaia} (\url{https://www.cosmos.esa.int/gaia}), processed by the {\it Gaia} Data
Processing and Analysis Consortium (DPAC,
\url{https://www.cosmos.esa.int/web/gaia/dpac/consortium}). Funding for the DPAC
has been provided by national institutions, in particular the institutions
participating in the {\it Gaia} Multilateral Agreement.  This research was
started at the NYC Gaia DR2 Workshop at the Center for Computational
Astrophysics of the Flatiron Institute in 2018 April.

AB acknowledges generous support from the Institute for Theory and Computation
at Harvard University.
% All code used in this work and all results are available at
% \url{https://github.com/adrn/GD1-DR2}.
}

\software{
    \package{Astropy} \citep{astropy, astropy:2018},
    \package{dustmaps}\footnote{\url{https://github.com/gregreen/dustmaps}},
    \package{gala} \citep{gala},
    \package{IPython} \citep{ipython},
    \package{matplotlib} \citep{mpl},
    \package{numpy} \citep{numpy},
    \package{scipy} \citep{scipy}
}

\bibliographystyle{aasjournal}
\bibliography{pal5fan}

\end{document}
