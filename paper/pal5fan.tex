% Copyright 2019 the authors. All rights reserved.

% TODO:
% -

\documentclass[modern]{aastex62}

\usepackage{amsmath}

% typography
\setlength{\parindent}{1.\baselineskip}
\newcommand{\acronym}[1]{{\small{#1}}}
\newcommand{\package}[1]{\textsl{#1}}
\newcommand{\gaia}{\textsl{Gaia}}
\newcommand{\pans}{\textsl{Pan-STARRS}}
\newcommand{\DR}{\acronym{DR2}}
\newcommand{\msun}{\textrm{M}_\odot}
\newcommand{\kpc}{\textrm{kpc}}
\newcommand{\kms}{\ensuremath{\textrm{km}~\textrm{s}^{-1}}}
\newcommand{\bs}[1]{\boldsymbol{#1}}
\newcommand{\masyr}{\ensuremath{\textrm{mas}~\textrm{yr}^{-1}}}
\newcommand{\feh}{\ensuremath{[\textrm{Fe} / \textrm{H}]}}
\newcommand{\given}{\,|\,}


\newcommand{\sectionname}{Section}
\newcommand{\equationname}{Equation}
\renewcommand{\tablename}{Table}
\usepackage{upgreek}

\newcommand{\todo}[1]{{\color{red} TODO: #1}}

\newcommand{\changes}[1]{{\textbf{#1}}}
% \newcommand{\changes}[1]{{#1}}

% aastex parameters
% \received{not yet; THIS IS A DRAFT}
%\revised{not yet}
%\accepted{not yet}
% % Adds "Submitted to " the arguement.
% \submitjournal{ApJ}
\shorttitle{Stuff}
\shortauthors{People}

%@arxiver{}

\begin{document}\sloppy\sloppypar\raggedbottom\frenchspacing % trust me

\title{Palomar 5's biggest fan}

% \author[0000-0003-0872-7098]{Adrian~M.~Price-Whelan}
% \affiliation{Department of Astrophysical Sciences,
%              Princeton University, Princeton, NJ 08544, USA}
% \email{adrn@astro.princeton.edu}
% \correspondingauthor{Adrian M. Price-Whelan}

% \author[0000-0002-7846-9787]{Ana Bonaca}
% \affiliation{Harvard--Smithsonian Center for Astrophysics, Cambridge, MA 02138, USA}

\begin{abstract}\noindent % trust me
    Words!
\end{abstract}

\keywords{Galaxy: halo --- dark matter ---
          Galaxy: kinematics and dynamics}

\section{Introduction}
\label{sec:intro}
\todo{1. Opening sentences on streams}.

\todo{2. Motivate deepest data of Pal 5}
\begin{itemize}
\item  Emphasize Pal 5 is the only cold stream for which we know the progenitor.
\item	 Mention other papers: Odenkirchen, \citet{Ibata:2016}, \citet{Ibata:2017}, \citet{Balbinot:2017}, Bernard (PS1: \citealt{Bernard:2016})
\item	 Mention remaining puzzles: main sequence stars, is the trailing arm thinner, does the leading arm truncate, are the gaps from the bar?
\item Maybe motivate through chaos, bar etc. that there might be crazy features we haven't seen. 
\item	We need deeper data, and data further south to fully understand what has created the gaps in Pal 5
\end{itemize}

	
%wrapping up intro: 
In this paper we present the deepest map of Pal 5 to date. \todo{one sentence on DECaLS)}. The data clearly show non-uniformities in both the leading and trailing arm of the stream. In particular, we discover a ``fanned" feature in the leading arm of the stream. We compare the data to a suite of simulated Pal 5 streams using a detailed gaussian mixture density model, and we explore mechanisms that could lead to the formation of the observed features.  

The paper is structured as follows: in Section \ref{sec:data}, we present new DECaLS Pal 5 data, and in Section \ref{sec:sim} we describe our simulation setups. We describe our Gaussian mixture model we use to fit the density and width of Pal 5 in Section \ref{sec:density}, and we show our comparisons of the simulations and data using in Section \ref{sec:results}. We discuss the implications of our results and address outstanding questions about Pal 5's morphology in Section \ref{sec:discussion}, and we conclude in Section \ref{sec:conclusion}. 



\section{Data}
\label{sec:data}

\begin{figure*}
\begin{center}
\includegraphics[width=0.95\textwidth]{fig1_a_map.pdf}
\includegraphics[width=0.95\textwidth]{fig1_b_cmds.pdf}
\end{center}
\caption{
(Top) The Legacy Surveys detection of the Palomar 5 globular cluster in a coordinate system aligned with its tidal tails.
(Bottom) Color-magnitude diagrams of $\approx0.8\times3$\,deg windows along the tails.
The regions are labeled in the top left of each panel, and their sky locations are marked in the top panel.
A stellar population consistent with Pal~5 is evident in every region, although its prominence varies between the fields.
There is a distance gradient along the stream, with the end of the trailing tail ($\phi_1\sim-15^\circ$) being the most distant, and end of the leading tail ($\phi_1\sim10^\circ$) the closest.
The fiducial Pal~5 isochrone is offset in every panel so that it matches the location of the main sequence, and the difference in distance modulus from the fiducial Pal~5 value is indicated in the top left.
}
\label{fig:cmds}
\end{figure*}



\section{Simulations}
\label{sec:sim}
In order to explore the mechanism leading the the observed morphology of the stream (i.e. the length asymmetry between the leading and trailing arm, as well as the gap in the trailing arm, and the ``fan" in the leading arm), we run a suite of Pal 5 simulations. In particular, we investigate whether Pal 5's ``fan" in the leading arm can be explained due to chaotic regions in the potential (\citealt{Pearson:2015}, \citealt{Price-Whelan:2016}, Yavetz et al., {\it in prep.}), or whether Pal 5 is interacting with the Galactic bar (\citealt{Pearson:2017}, \citealt{Erkal:2017}, \citealt{Banik:2019}). In Section \ref{sec:potential}, we describe the potentials we use to simulate the evolution of Pal 5, and in Section \ref{sec:potential} we describe our simulation setup. 

\subsection{Potential}
\label{sec:potential}
We simulate the evolution of Pal 5 in two classes of three-component Galactic potentials: 

\begin{itemize}
\item[1.] {\bf Static potential}: we use the {\small MWPotential2014} (\citealt{Bovy:2015}) consisting of a Miyamoto-Nagai disk (\citealt{Miyamoto:1975}), a bulge modeled as an exponentially cut off, power-law density profile, and an NFW dark matter halo (\citealt{Navarro:1996}). We vary the flattening of the NFW halo ($q_z = 0.94$ and $q_z = 0.5$) to investigate Pal 5's morphology in on both a regular and chaotic orbit (see Section \ref{sec:modeling}). 

\item[2.] {\bf  Barred potential}: we use the same disk and halo as in {\small MWPotential2014}, but include a Galactic bar instead of a bulge. Following \citet{wang:2012}, we compute the bar potential as a basis-function expansion (BFE) representation of a triaxial, exponential density profile:

\begin{equation}
\rho_{bar} = \rho_0 [{\rm exp} (-r^2_1/2) + r_2^{-1.85} {\rm exp}(-r_2) ]
\end{equation}

\begin{equation}
r_1 = \left[\left((x/x_0)^2 + (y/y_0)^2\right)^2 +( z/z_0)^4\right]^{1/4}
\end{equation}

\begin{equation}
r_2 = \left[\frac{q^2(x^2 + y^2) + z^2)}{z_0^2}\right]^{1/2}
\end{equation}
where the scale lengths are $x_0$ = 1.49 kpc, $y_0$ = 0.58 kpc, $z_0$ = 0.4 kpc, and q = 0.6. We include terms up to $n=9$, $l=19$ in the ``self-consistent field" BFE formalism, as this yields a good representation of the density of the bar (\citealt{Banik:2019})\footnote{Note that using lower order terms (e.g. n=6, l = 8) for the basis function expansion does not much change the morphology or kinematics of the Pal 5 stream.}. We explore barred models with pattern speeds of $\Omega_b$ = ($25 - 65$) $\kms$ kpc$^{-1}$ in increments of 1 kpc$^{-1}$, and  we vary the bar mass to be $M_{bar} = 5 \times 10^{9}$ $\msun$ or $M_{bar} = 1 \times 10^{10}$ $\msun$ (\citealt{Portail:2017}). Additionally, we fix the present day angular offset from the Galactic x-axis in the direction of rotation, $\alpha = 27\deg$.
\end{itemize}

\citet{wang:2012} construct a bar with a pattern speed of $\Omega_b$ =  60 $\kms$ kpc$^{-1}$, which has a co-rotation radius, $r_{\rm CR} = 3.7$ kpc. However, in this paper, we explore a range of pattern speeds, and this will change the co-rotation radii (faster bars have co-rotation at smaller radii and vice versa).  As bars are not expected to extend beyond their co-rotation radius (\citealt{binney:2008}), we therefore adjust the physical scaling of the bar when varying the pattern speed. %In \citet{wang:2012}, the scale-radius, $r_s$, assumed in the BFE, is  $r_s = 1.1$ kpc. 
In particular, we  compute the co-rotation radius, $r_{\rm CR},$ based on the mass profiles of the static potential at any given pattern speed, $\Omega_b$. We then scale our bar model for a given pattern speed, $\Omega_b$, by:

\begin{equation} 
r_{s, \Omega_b}  = r_{{\rm CR}, \Omega_b}/r_{{\rm CR, Wang 2012}}
\end{equation} 

If this scaling is not included (as for bar models used in \citealt{price:2016b}, \citealt{Pearson:2017}, \citealt{Erkal:2017}, \citealt{Banik:2019}), this creates a too strong bar quadrupole for the faster pattern speeds, and a too weak bar quadrupole for the slower pattern speeds. 

In Figure \ref{fig:vcirc}, we show the circular velocity as a function of radius for the barred, three-component Galactic potential with $\Omega_b$ = ($25 - 65$) $\kms$ kpc$^{-1}$. The red lines correspond to the potentials with a bar mass of $M_{bar} = 5 \times 10^{9}$ $\msun$, and the blue lines correspond to potentials with a bar mass of  $M_{bar} = 1 \times 10^{10}$ $\msun$.

\begin{figure}
\centerline{\includegraphics[width=\columnwidth]{v_circ_nlm919.png}}
\caption{\todo{Probably include static potential as well, and show the different components of the potentials }}
\label{fig:vcirc}
\end{figure}



\subsection{Stream modeling}
\label{sec:modeling}
To simulate the evolution of Pal 5 in a given potential, we first transform Pal 5's observed 6D phase space coordinates (see Table \ref{tab:pal5}) into a Galactocentric frame by assuming $v_{lsr} = (11.1, 24.0, 7.25) \kms$,  $v_{circ} = 220  \kms$ and a distance from the Sun to the Galactic centre of 8.1 kpc. 

%Put into table 
%(RA = 229.018 deg, Dec = - 0.24 deg, distance = 22.9 kpc, $v_r$ = -58.7 , pm$_{RA,cosdec}= -2.296$ mas and pm$_{Dec} = -2.257$ mas

We then integrate the cluster backwards in time for 4 Gyr in steps of 0.5 Myr in a given potential. Subsequently, we simulate the forward evolution of Pal 5 using the ``particle-spray" stream generating method developed by \citet{Fardal:2015}, such that the cluster ends at its present day position at $t = 0$. We release two particles through each of the two Lagrange points every 10 Myr. 

To investigate Pal 5's evolution on a regular orbit, we first run a ``particle-spray" simulations setting $q_z = 0.94$ (\citealt{Bovy:2016}). Subsequently, to we run the same simulation setting $q_z = 0.5$ to induce a chaotic orbit. 

Additionally, we simulate the evolution of Pal 5's stream in the barred potential, where we first integrating the orbit backwards for 4 Gyr. We then create mock streams using the ``particle-spray" method. We vary the pattern speed of the bar, while updating its physical scaling (see Section \ref{sec:potential} and Figure \ref{fig:vcirc}). We do this exercise with a bar mass of $M_{bar} = 5 \times 10^{9}$ $\msun$ and $M_{bar} = 1 \times 10^{10}$ $\msun$.

\subsection{Simulation results}
\label{sec:sim_results}
In this Section, we show the results of modeling Pal 5 in a static potentail with varying halo flattening (Section \ref{sec:results_stat}), and in a barred potentials with various pattern speeds (Section \ref{sec:results_barred}). In both cases, we compare the simulations to our data using the density model described in Section \ref{sec:density}.


%\label{sec:results_stat}
%The left panel of Figure \ref{fig:sims} shows the Pal 5 on a regular orbit in a static potential with a flattening of $q_z = 0.94$ (\citealt{bovy:2017}). The leading and trailing arm of Pal 5 are symmetric, as expected (e.g. \citealt{dehnen:2004}, \citealt{Pearson:2015}). In the right panel, we show a stream evolved in a flattened static potential ($q_z = 0.5$). For both scenarios the leading and trailing arm look symmetric, although the arms are ``fanned" out in the chaotic case (\citealt{Price-Whelan:2016}).
%
%
%\label{sec:results_barred}
%In Figure \ref{fig:sims} we show four different simulated Pal 5 streams that all yield densities and widths comparable to the data. These streams were evolved in a barred potential with pattern speeds of $\Omega_b = (...) \kms$ kpc$^{-1}$. We selected these four examples through visual inspection of the density and width of the data and the simulated streams. 



\begin{figure}
\centerline{\includegraphics[width=\columnwidth]{bar.png}}
\caption{\todo{Placeholder: have two columns with 1) morpohology of sims 2) radial velocities, black colors. }}
\label{fig:sims}
\end{figure}


\section{Density model}
\label{sec:density}
To compare our data and various Pal 5 simulated streams, we construct density model consisting of a multicomponent Gaussian mixture model. In particular, we are interested in fitting the width and density of our data and simulated stars.  

We first transform our simulated Pal 5 data points to the tangent sky plane using a Zeanit (Lambert azimuthal) equal-area projection, such that we can define a Gaussian. We call these coordinates, $X$, $Y$.

We then fit a 3rd order polynomial to the leading and trailing arm separately, in this projected space. 

We place K nodes, k, equally in distance along the polynomial fits to the leading and trailing arm. 

At each node, k, along the polynomial we find the tangent/parallel unit vector,  $\hat{u}$, and the perpendicular normal unit vector, $\hat{v}$, to the stream. 

At each node, k, we define the co-variance matrix, $\tilde{C_k}$:

\begin{equation}
\tilde{C_k} = 
\begin{pmatrix}
    h^2 & 0  \\
    0 & S_k^2  \\
\end{pmatrix}
\end{equation}
where $h$ is bandwidth of the Gaussian components along the polynomial fit and $S_k$ is the  width of the Gaussians in the perpendicular (normal) direction of the stream at any node, k. 

We transform it from the space spanned by the $\hat{u}$,  $\hat{v}$ vectors to $X$, $Y$:
\begin{equation}
C_k = \rm{R} \tilde{C_k} \rm{R}^T
\end{equation}
where
\begin{equation}
\rm{R} = 
\begin{pmatrix}
    \rm{cos} \theta & - \rm{sin} \theta  \\
    \rm{sin} \theta & \rm{cos} \theta \\
\end{pmatrix}
\end{equation}
and  $\theta$ is the angle between the tangent sky plane and the unit vector, $\hat{u}$.

To compute our density model along the leading at trailing stream, for the K nodes, k, we define ln density  and sum over the K nodes:
\begin{equation}
\rm{ln} \sum ( \alpha_k \mathcal{N}\left(\mu_k, C_k\right))
\end{equation}
where $\mu_k$ denotes the location of each node, k, along the leading and trailing polynomial fits, respectively, $C_k$ is the covariance matrix, and $\alpha_k$ is the amplitude of the Gaussian (representing density at specific node, k). 

We then fit for the perpendicular width, $S_k$, to the stream (polynomial fits) and the amplitude of the Gaussians, $\alpha_k$, which represent the density along the stream. 

%Explain background model for data.

We compute the width and density of both or simulated Pal 5 streams and the DECaLS data fitting the above Gaussian mixture model. 




\section{Discussion}
\label{sec:discussion}

\subsection{Deeper stream data}
Do all streams look crazy if we get deeper data? Discuss ``stream-fanning", bar, VL2 Banaca paper. 

\subsection{Other perturbers}
As Pal 5 is moving prograde with respect to the disk it will be subject to interactions with both the bar (\citealt{Hattori:2016}, \citealt{price:2016b}), molecular clouds (\citealt{Amorisco:2016}) and possibly spiral arms (\citealt{Banik:2019}). Additionally, dark matter substructure could be interacting with the stream. Pal 5 apocentric distance is $\sim 18$ kpc, and the cluster is therefore probing the inner part of the Galatic potential. Hence, it might be unlikely that there are many dark matter subhalos in this part of the halo (\citealt{Garrison-Kimmel:2017}). However, the GD1 stellar stream orbit probes a similar region of the Galatic potential, and shows evidence of an interaction with a dark substructure (\citealt{Price-Whelan:2018}, \citealt{Bonaca:2018b}) as the data shows a gap and a ``spur" (\citealt{Yoon:2011}). 



\section{Conclusion}
\label{sec:conclusion}


\acknowledgements{
It is a pleasure to thank


\appendix
\section{Math for density model}
We define our likelihood function, $\mathcal{L} $, as:

\begin{equation}
    \begin{split}
     \mathcal{L} = P(\{x_n\} \given \{\mu_k\}, \{C_k\}, \{\alpha_k\} ) &= 
            \prod_n \sum_k P(x_n \given \mu_k, C_k, \alpha_k )  \quad 
    \end{split}
\end{equation}
where $K$ is the total number of nodes, $\mu_k$ is the mean position of each node, $N$ is the total number of ``data points", $\alpha$ is the amplitude, $x_n$ = $\begin{pmatrix} x\\  y \end{pmatrix}$ is the data points and $P$ is:
\begin{equation}
    \begin{split}
       P(x_n \given \mu_k, C_k, \alpha_k ) = \alpha_k \mathcal{N}(x_n \given \mu_k, C_k)  \quad 
    \end{split}
\end{equation}

$C_k$ is the covariance matrix in the $X,Y$-space which is the tangent plane sky projection. We transform to $X,Y$-space from the space spanned by the $\hat{u}$,  $\hat{v}$ vectors:
\begin{equation}
C_k^{(x,y)} = \rm{R} \tilde{C_k} \rm{R}^{\rm T}
\end{equation}
where
\begin{equation}
\tilde{C_k}^{(\hat{u},\hat{v})} = 
\begin{pmatrix}
    {\rm h}^2 & 0  \\
    0 & S_k^2  \\
\end{pmatrix}
\end{equation}
and where h is bandwidth of the Gaussian components (which we fix) along the polynomial fit, and $S_k$ is the  width of the Gaussians in the perpendicular (normal) direction of the stream at any node, k. R is the rotation matrix from $\hat{u}$,  $\hat{v}$ to $X, Y$-space:

\begin{equation}
\rm{R} = 
\begin{pmatrix}
    \rm{cos} \theta & - \rm{sin} \theta  \\
    \rm{sin} \theta & \rm{cos} \theta \\
\end{pmatrix}
\end{equation}
and  $\theta$ is the angle between the tangent sky plane and the unit vector, $\hat{u}$.

To optimize for the likelihoods, we compute the analytic derivatives of the log likelihoods, ln$\mathcal{L}$:
 \begin{equation}
 \begin{split}
    {\rm ln} \mathcal{L} = \sum_n {\rm ln} \left[\sum_k P(x_n \given \mu_k, C_k, \alpha_k)\right]  \quad 
    \end{split}
\end{equation}
hence, we want to compute:
 \begin{equation}
 \begin{split}
    \frac{\partial{\rm ln} \mathcal{L}}{\partial \alpha_k}, \frac{\partial{\rm ln} \mathcal{L}}{\partial \mu_k}, \frac{\partial{\rm ln} \mathcal{L}}{\partial S_k}  \quad 
    \end{split}
\end{equation}

To do so, we need to use the following:
 \begin{equation}
 \begin{split}
 |C_k^{-1}|^{1/2} = \frac{1}{{\rm h} S_k},
   \end{split}
\end{equation}

 \begin{equation}
 \begin{split}
  \alpha_K = 1 - \sum_k^{K-1} \alpha_k,
   \end{split}
\end{equation}

 \begin{equation}
 \begin{split}
 \mathcal{N}(x_n \given \mu_k, C_k) = \frac{1}{2\pi}|C_k^{-1}|^{1/2} {\rm exp}\left[-\frac{1}{2}(x_n - \mu_k)^{\rm T} C_k^{-1} (x_n - \mu_k) \right]  \quad 
   \end{split}
\end{equation}

We can now compute the partial derivatives of the log likelihoods, $\partial$ln$\mathcal{L}$:
 \begin{equation}
 \begin{split}
    \frac{\partial{\rm ln} \mathcal{L}}{\partial \alpha_k}  = \sum_n \frac{ \mathcal{N}(x_n \given \mu_k, C_k) -  \mathcal{N}(x_n \given \mu_K, C_K)}{\sum_k \alpha_K \mathcal{N}(x_n \given \mu_K, C_K)} \quad 
    \end{split}
\end{equation}
\todo{check capital K's above}
and

 \begin{equation}
 \begin{split}
\frac{\partial{\rm ln} \mathcal{L}}{\partial \mu_k} = \sum_n \frac{\alpha_k \mathcal{N}(x_n \given \mu_k, C_k) C_k^{-1} (x_n - \mu_k)}{\sum_k \alpha_K \mathcal{N}(x_n \given \mu_K, C_K)}
    \end{split}
\end{equation}
and

 \begin{equation}
 \begin{split}
\frac{\partial{\rm ln} \mathcal{L}}{\partial S_k} = \sum_n \frac{\alpha_k \mathcal{N}(x_n \given \mu_k, C_k)}{\sum_k \alpha_K \mathcal{N}(x_n \given \mu_K, C_K)} \left[ \frac{1}{S_k} \left(\frac{1}{S_k^2} \left( \frac{1}{b_2} {\rm cos \theta} -  \frac{1}{b_1} {\rm sin} \theta \right)^2 - 1\right)\right]
    \end{split}
\end{equation}
where
$\begin{pmatrix} b_1\\  b_2 \end{pmatrix}= x_n - \mu_k$. 

%\begin{equation}
%    \begin{split}
%    p(\phi_2 \given \bs{\theta}) &=
%            \alpha_{\textrm{bg}} \, \mathcal{U}(-10, 5) \\ & \quad +
%            \alpha_{\textrm{s}, 1} \, \mathcal{N}(\phi_2 \given \mu_{\textrm{s}}, \sigma_{\textrm{s}, 1}) +
%            \alpha_{\textrm{s}, 2} \, \mathcal{N}(\phi_2 \given \mu_{\textrm{s}}, \sigma_{\textrm{s}, 2}) \\ & \quad +
%            \alpha_{\textrm{f}} \,
%                \mathcal{N}(\phi_2 \given \mu_{\textrm{f}}, \sigma_{\textrm{f}})
%    \end{split}
%\end{equation}


% 
% This work has made use of data from the European Space Agency (ESA) mission {\it
% Gaia} (\url{https://www.cosmos.esa.int/gaia}), processed by the {\it Gaia} Data
% Processing and Analysis Consortium (DPAC,
% \url{https://www.cosmos.esa.int/web/gaia/dpac/consortium}). Funding for the DPAC
% has been provided by national institutions, in particular the institutions
% participating in the {\it Gaia} Multilateral Agreement.  This research was
% started at the NYC Gaia DR2 Workshop at the Center for Computational
% Astrophysics of the Flatiron Institute in 2018 April.
% 
% AB acknowledges generous support from the Institute for Theory and Computation
% at Harvard University.
% All code used in this work and all results are available at
% \url{https://github.com/adrn/GD1-DR2}.
}

\software{
    \package{Astropy} \citep{astropy, astropy:2018},
    \package{dustmaps}\footnote{\url{https://github.com/gregreen/dustmaps}},
    \package{gala} \citep{gala},
    \package{IPython} \citep{ipython},
    \package{matplotlib} \citep{mpl},
    \package{numpy} \citep{numpy},
    \package{scipy} \citep{scipy}
}

\bibliographystyle{aasjournal}
\bibliography{pal5fan}

\end{document}
